\documentclass{tufte-book}
\hypersetup{colorlinks}
%% metadata
\title{Desiging Indoor model Airplanes\\Elementary Aerodynamics}
\author[www.co-pylit-org/math-magik]{Roie R. Black}
\publisher{Roie R. Black}

% typeset tabular material
\usepackage{booktabs}

% for graphics images
\usepackage{graphicx}
\setkeys{Gin}{width=\linewidth,totalheight=\textheight,keepaspectratio}
\graphicspath{{assets/images/}}

% The fancyvrb package lets us customize the formatting of verbatim
% environments.  We use a slightly smaller font.
\usepackage{fancyvrb}
\fvset{fontsize=\normalsize}

%%
% Prints argument within hanging parentheses (i.e., parentheses that take
% up no horizontal space).  Useful in tabular environments.
\newcommand{\hangp}[1]{\makebox[0pt][r]{(}#1\makebox[0pt][l]{)}}

%%
% Prints an asterisk that takes up no horizontal space.
% Useful in tabular environments.
\newcommand{\hangstar}{\makebox[0pt][l]{*}}

%%
% Prints a trailing space in a smart way.
\usepackage{xspace}

%%
% Some shortcuts for Tufte's book titles.  The lowercase commands will
% produce the initials of the book title in italics.  The all-caps commands
% will print out the full title of the book in italics.
\newcommand{\vdqi}{\textit{VDQI}\xspace}
\newcommand{\ei}{\textit{EI}\xspace}
\newcommand{\ve}{\textit{VE}\xspace}
\newcommand{\be}{\textit{BE}\xspace}
\newcommand{\VDQI}{\textit{The Visual Display of Quantitative Information}\xspace}
\newcommand{\EI}{\textit{Envisioning Information}\xspace}
\newcommand{\VE}{\textit{Visual Explanations}\xspace}
\newcommand{\BE}{\textit{Beautiful Evidence}\xspace}

\newcommand{\TL}{Tufte-\LaTeX\xspace}

% Prints the month name (e.g., January) and the year (e.g., 2008)
\newcommand{\monthyear}{%
  \ifcase\month\or January\or February\or March\or April\or May\or June\or
  July\or August\or September\or October\or November\or
  December\fi\space\number\year
}


% Prints an epigraph and speaker in sans serif, all-caps type.
\newcommand{\openepigraph}[2]{%
  %\sffamily\fontsize{14}{16}\selectfont
  \begin{fullwidth}
  \sffamily\large
  \begin{doublespace}
  \noindent\allcaps{#1}\\% epigraph
  \noindent\allcaps{#2}% author
  \end{doublespace}
  \end{fullwidth}
}

% Inserts a blank page
\newcommand{\blankpage}{\newpage\hbox{}\thispagestyle{empty}\newpage}

\usepackage{units}

% Typesets the font size, leading, and measure in the form of 10/12x26 pc.
\newcommand{\measure}[3]{#1/#2$\times$\unit[#3]{pc}}

% Macros for typesetting the documentation
\newcommand{\hlred}[1]{\textcolor{Maroon}{#1}}% prints in red
\newcommand{\hangleft}[1]{\makebox[0pt][r]{#1}}
\newcommand{\hairsp}{\hspace{1pt}}% hair space
\newcommand{\hquad}{\hskip0.5em\relax}% half quad space
\newcommand{\TODO}{\textcolor{red}{\bf TODO!}\xspace}
\newcommand{\na}{\quad--}% used in tables for N/A cells
\providecommand{\XeLaTeX}{X\lower.5ex\hbox{\kern-0.15em\reflectbox{E}}\kern-0.1em\LaTeX}
\newcommand{\tXeLaTeX}{\XeLaTeX\index{XeLaTeX@\protect\XeLaTeX}}
% \index{\texttt{\textbackslash xyz}@\hangleft{\texttt{\textbackslash}}\texttt{xyz}}
\newcommand{\tuftebs}{\symbol{'134}}% a backslash in tt type in OT1/T1
\newcommand{\doccmdnoindex}[2][]{\texttt{\tuftebs#2}}% command name -- adds backslash automatically (and doesn't add cmd to the index)
\newcommand{\doccmddef}[2][]{%
  \hlred{\texttt{\tuftebs#2}}\label{cmd:#2}%
  \ifthenelse{\isempty{#1}}%
    {% add the command to the index
      \index{#2 command@\protect\hangleft{\texttt{\tuftebs}}\texttt{#2}}% command name
    }%
    {% add the command and package to the index
      \index{#2 command@\protect\hangleft{\texttt{\tuftebs}}\texttt{#2} (\texttt{#1} package)}% command name
      \index{#1 package@\texttt{#1} package}\index{packages!#1@\texttt{#1}}% package name
    }%
}% command name -- adds backslash automatically
\newcommand{\doccmd}[2][]{%
  \texttt{\tuftebs#2}%
  \ifthenelse{\isempty{#1}}%
    {% add the command to the index
      \index{#2 command@\protect\hangleft{\texttt{\tuftebs}}\texttt{#2}}% command name
    }%
    {% add the command and package to the index
      \index{#2 command@\protect\hangleft{\texttt{\tuftebs}}\texttt{#2} (\texttt{#1} package)}% command name
      \index{#1 package@\texttt{#1} package}\index{packages!#1@\texttt{#1}}% package name
    }%
}% command name -- adds backslash automatically
\newcommand{\docopt}[1]{\ensuremath{\langle}\textrm{\textit{#1}}\ensuremath{\rangle}}% optional command argument
\newcommand{\docarg}[1]{\textrm{\textit{#1}}}% (required) command argument
\newenvironment{docspec}{\begin{quotation}\ttfamily\parskip0pt\parindent0pt\ignorespaces}{\end{quotation}}% command specification environment
\newcommand{\docenv}[1]{\texttt{#1}\index{#1 environment@\texttt{#1} environment}\index{environments!#1@\texttt{#1}}}% environment name
\newcommand{\docenvdef}[1]{\hlred{\texttt{#1}}\label{env:#1}\index{#1 environment@\texttt{#1} environment}\index{environments!#1@\texttt{#1}}}% environment name
\newcommand{\docpkg}[1]{\texttt{#1}\index{#1 package@\texttt{#1} package}\index{packages!#1@\texttt{#1}}}% package name
\newcommand{\doccls}[1]{\texttt{#1}}% document class name
\newcommand{\docclsopt}[1]{\texttt{#1}\index{#1 class option@\texttt{#1} class option}\index{class options!#1@\texttt{#1}}}% document class option name
\newcommand{\docclsoptdef}[1]{\hlred{\texttt{#1}}\label{clsopt:#1}\index{#1 class option@\texttt{#1} class option}\index{class options!#1@\texttt{#1}}}% document class option name defined
\newcommand{\docmsg}[2]{\bigskip\begin{fullwidth}\noindent\ttfamily#1\end{fullwidth}\medskip\par\noindent#2}
\newcommand{\docfilehook}[2]{\texttt{#1}\index{file hooks!#2}\index{#1@\texttt{#1}}}
\newcommand{\doccounter}[1]{\texttt{#1}\index{#1 counter@\texttt{#1} counter}}

% Generates the index
\usepackage{makeidx}
\makeindex


\begin{document}
\frontmatter
\blankpage
\newpage\thispagestyle{empty}
\vfill
\openepigraph{%
A designer knows that he has achieved perfection 
not when there is nothing left to add, 
but when there is nothing left to take away.
}{Antoine de Saint-Exup\'{e}ry}
\vfill

\maketitle

\newpage
\begin{fullwidth}
~\vfill
\thispagestyle{empty}
\setlength{\parindent}{0pt}
\setlength{\parskip}{\baselineskip}
Copyright \copyright\ \the\year\ \thanklessauthor

\par\smallcaps{Published by \thanklesspublisher}

\par\smallcaps{tufte-latex.github.io/tufte-latex/}

\par Licensed under the Apache License, Version 2.0 (the ``License''); you may not
use this file except in compliance with the License. You may obtain a copy
of the License at \url{http://www.apache.org/licenses/LICENSE-2.0}. Unless
required by applicable law or agreed to in writing, software distributed
under the License is distributed on an \smallcaps{``AS IS'' BASIS, WITHOUT
WARRANTIES OR CONDITIONS OF ANY KIND}, either express or implied. See the
License for the specific language governing permissions and limitations
under the License.\index{license}

\par\textit{First printing, \monthyear}
\end{fullwidth}

% r.5 contents
\tableofcontents

\listoffigures

\listoftables

% r.7 dedication
\cleardoublepage
~\vfill
\begin{doublespace}
\noindent\fontsize{18}{22}\selectfont\itshape
\nohyphenation
Dedicated to those who appreciate \LaTeX{} 
and the work of \mbox{Edward R.~Tufte} 
and \mbox{Donald E.~Knuth}.
\end{doublespace}
\vfill
\vfill


% r.9 introduction
\cleardoublepage
\chapter*{Introduction}

This sample book discusses the design of Edward Tufte's
books\cite{Tufte2001,Tufte1990,Tufte1997,Tufte2006}
and the use of the \doccls{tufte-book} and \doccls{tufte-handout} document classes.


%%
% Start the main matter (normal chapters)
\mainmatter


\chapter{The Design of Tufte's Books}
\label{ch:tufte-design}


\newthought{The pages} of a book are usually divided into three major
sections: the front matter (also called preliminary matter or prelim), the
main matter (the core text of the book), and the back matter (or end
matter).

\newthought{The front matter} of a book refers to all of the material that
comes before the main text.  The following table from shows a list of
material that appears in the front matter of \VDQI, \EI, \VE, and \BE
along with its page number.  Page numbers that appear in parentheses refer
to folios that do not have a printed page number (but they are still
counted in the page number sequence).

\bigskip
\begin{minipage}{\textwidth}
\begin{center}
\begin{tabular}{lcccc}
\toprule
 & \multicolumn{4}{c}{Books} \\
\cmidrule(l){2-5} 
Page content & \vdqi & \ei & \ve & \be \\
\midrule
Blank half title page & \hangp{1} & \hangp{1} & \hangp{1} & \hangp{1} \\
Frontispiece\footnotemark{}
  & \hangp{2} & \hangp{2} & \hangp{2} & \hangp{2} \\
Full title page & \hangp{3} & \hangp{3} & \hangp{3} & \hangp{3} \\
Copyright page & \hangp{4} & \hangp{4} & \hangp{4} & \hangp{4} \\
Contents & \hangp{5} & \hangp{5} & \hangp{5} & \hangp{5} \\
%Blank & -- & \hangp{6} & \hangp{6} & \hangp{6} \\
Dedication & \hangp{6} & \hangp{7} & \hangp{7} & 7 \\
%Blank & -- & \hangp{8} & -- & \hangp{8} \\
Epigraph & -- & -- & \hangp{8} & -- \\
Introduction & \hangp{7} & \hangp{9} & \hangp{9} & 9 \\
\bottomrule
\end{tabular}
\end{center}
\end{minipage}
\vspace{-7\baselineskip}\footnotetext{The contents of this page vary from book to book.  In
  \vdqi this page is blank; in \ei and \ve this page holds a frontispiece;
  and in \be this page contains three epigraphs.}
\vspace{7\baselineskip}

\bigskip
The design of the front matter in Tufte's books varies slightly from the
traditional design of front matter.  First, the pages in front matter are
traditionally numbered with lowercase roman numerals (e.g., i, ii, iii,
iv,~\ldots).  Second, the front matter page numbering sequence is usually
separate from the main matter page numbering.  That is, the page numbers
restart at 1 when the main matter begins.  In contrast, Tufte has
enumerated his pages with arabic numerals that share the same page counting
sequence as the main matter.  

%%
% The back matter contains appendices, bibliographies, indices, glossaries, etc.r

\backmatter

\bibliography{assets/references}
\bibliographystyle{plainnat}


\printindex

\end{document}

